\documentclass[12pt]{article}
\title{Chapter 2 Exercises}
\author{Blake Gateley}
\date{December 6th, 2023}
\usepackage{graphicx}

\pagenumbering{arabic}

\begin{document}

\begin{titlepage}
   \begin{center}
       \vspace*{1cm}

       {\Huge \textbf{Fundamentals of Discrete Mathematics}}

       \vspace{0.5cm}
        % subtitle
            
       \vspace{1.5cm}

       \textbf{Blake Gateley}


       
       \vfill
            
       An in depth look into discrete mathematics
            
       \vspace{0.8cm}
     
   
            
       Discrete Mathematics\\
       Allen High School\\
       12/06/2023
            
   \end{center}
\end{titlepage}
\tableofcontents

\newpage
%----------------------------------------------
10.1 - Five of the following statements are negations of the other five. Pair each statement with its negation. 

c and f, a and h, b and g, d and e, i and j     

 (a) - 0 1 1 0 - p \oplus q
 (b) - 0 1 0 0 - \neg p \wedge q
 (c) - 1 1 1 1 - p \Rightarrow (q \Rightarrow p)
 (d) - 1 1 0 1 - p \Rightarrow q
 (e) - 0 0 1 0 - p \wedge \neg q
 (f) - 0 0 0 0 - q \wedge (p \wedge \neg p)
 (g) - 0 0 1 0 - p \vee \neg q
 (h) - 1 0 0 1 - p \Leftrightarrow q
 (i) - 0 0 1 1  - p \wedge (q \vee \neg q)
 (j) - 1 1 0 0 - (p \Rightarrow q) \Rightarrow p

 10.3 - For each of the following formulas, decide whether it is a tautology, satisfiable, or unsatisfiable. Justify your answers
	
 (a) - 1 1 1 1 - Tautology - (p \vee q)\vee(q \Rightarrow p)
 (b) - 0 0 1 1 - Satisfiable - (p \Rightarrow q) \Rightarrow p
 (c) - 0 1 1 1 - Satisfiable - p \Rightarrow (q \Rightarrow p)
 (d) - 0 0 0 0 - Unsatisfiable - (\neg p \wedge q)\wedge(q \Rightarrow p)
 (e) - 1 0 1 1 - Satisfiable - (p \Rightarrow q)\Rightarrow(\neg p \Rightarrow \neg q)
 (f) - 1 1 1 1     Tautology - (\neg p \Rightarrow \neg q)\Leftrightarrow(q \Rightarrow p)

 10.5 - (a) Show that for any formulas a, b, and y 
 (\alpha \wedge \beta)\vee \alpha \vee \gamma \equiv \alpha \vee \gamma

 (b)Give the corresponding rule for simplifying
 (\alpha \vee \beta)\wedge \alpha \wedge \gamma 
 (c)Find the simplest possible disjunctive and conjunctive normal forms of the formula
 (p \wedge q)\Rightarrow(p \oplus q)



 10.7 - In this problem you will show that putting a formula into conjunctive normal form may increase its length exponentially. Consider the formula

 where n≥1 and the pi and qi are propositional variables. This formula has length 4n − 1, if we count each occurrence of a propositional variable or  an operator as adding 1 to the length and ignore the implicit parentheses.

 (a) Write out a conjunctive normal form of this formula in the case n = 3
 (b) How long is your conjunctive normal form of this formula, using the same conventions as above? For general n, how long is the conjunctive normal form as a function of n?
 (c) Show that similarly, putting a formula into disjunctive normal form may                                                                           
 increase its length exponentially.
 (d) Consider the following algorithm for determining whether a formula                                                                                
 is satisfiable: Put the formula into disjunctive normal form by the method of this chapter, and then check to see if all of the disjuncts are contradictions (containing both a variable and its complement). If not, the formula is satisfiable. Why is this algorithm exponentially costly?

 10.9 - This problem introduces resolution theorem-proving.

 \end{document}
